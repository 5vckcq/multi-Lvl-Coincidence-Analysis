% evtl. hilfreich:
% - Subgraph (vgl. 28.10.2 in pgf Handbuch)
\documentclass[11pt]{standalone}
\usepackage{tikz}
\usetikzlibrary{arrows, backgrounds, fit, positioning, shapes, graphs, quotes, calc} %%, graphdrawing}
%%\usegdlibrary{layered}

\tikzset{>=latex}

\begin{document}
\def\hilightsource#1{\draw[densely dashed,-latex] ($(#1)-(8mm,0)$) -> ($(#1)-(3mm,0)$);
  \fill [green, opacity=.25] (#1) circle [radius=3mm]; }
\def\hilighttarget#1{\draw[densely dashed,-latex] ($(#1)+(3mm,0)$) -> ($(#1)+(8mm,0)$);
  \fill [red, opacity=.25] (#1) circle [radius=3mm]; }

\tikz[every node/.style={circle, draw, minimum size=6mm, text width=5.5mm, align=center, inner sep = 0pt},
  line join=round,
  every new -!-/.style = {width=211mm},
  every new --/.style = {dotted,thin,bend left=20},
  every new ->/.style = {densely dotted},
  every new <->/.style = {densely dotted}]
\graph[math nodes,
  grow right=20mm, branch up = 14mm, 
%layered layout,
%%grow=up, components go right,
%%level distance=18mm, sibling sep=1mm, sibling distance=12mm,
  operator=\tikzgraphforeachcolorednode{source}{\hilightsource},
  operator=\tikzgraphforeachcolorednode{target}{\hilighttarget},
]
%{  A->{{B,C},D->E}, F->{D->E,G}, I->J->K->L }; % Beispiel 1
%{k->l,{{c->d->{,h->i,e->f},a}->j},a_1->a_2};   % Beispiel 2
%{ % Beispiel 3
%{E_1->{E_3,E_2}->E_4-!-E_5->E_6->E_7,E_5->E_6->E_7,E_1->E_2->E_4}; % E_4-!-E_5 um beide Ketten auf eine Ebene zu setzen, am Ende E_1->E_2->E_4 um E_4 korrekterweise als Ende
%{{{F_1,F_2}->F_4,F_3->{F_6,F_5}}->F_7}; % zweite Ebene
%{-!-G-!-};  % obere Ebene
%{-!-(E_1.west)--(F_1.west)-!-,-!-(F_1.east)--(E_4.east)-!-,-!-(E_5.west)--(F_7.west)-!-,-!-(F_7.east)--(E_7.east)-!-,-!-(G.east)--(F_7.east)-!-,-!-(F_1.west)--(G.west)-!-};
%}; % Beispiel 3 Ende
%%{ [same layer] F_1, F_7 };                    % Beispiel 3 fuer layered layout
%%{ [same layer] E_1, E_4, E_5, E_7 };};
{ % Beispiel 4
{OT<->other_1}; % molecular
{DE->CB->AX}; % sub-cellular
{OVC->CJC<->CVC}; % cell
{RS->{CCS,SCS}-!-OFC<->{VPC,other_2}}; % brain region
{{EXD,RA,IND,MtC}->{OV_1,OV_2}->CO,{EXD,RA,IND,MtC}->{OV_2,OV_1}->CO}; % good based model
{-!-(DE.west)--CJC--(AX.east)-!-,-!-OT--RS-!-,-!-RS--OT-!-,-!-VPC--CCS-!-,-!-(CO.east)--(VPC.east)-!-,-!-(OFC.west)--(OV_1.west)-!-}; % zu Anfang und Ende "-!-" um keine neuen source- bzw. target-nodes zu setzen, ausserdem: (X.east) bzw. (X.west) sorgt dafür, dass Interlevelverbindungen aussen ansetzen Kausalverbindungen; Verbindungen wiederholen um korrekte source- und target-nodes zu bekommen
{OT<->other_1}; % molecular
{DE->CB->AX}; % sub-cellular
{OVC->CJC<->CVC}; % cell
{OFC<->VPC,OFC<->other_2,RS->{CCS,SCS}}; % brain region
{{EXD,RA,IND,MtC}->{OV_1,OV_2}->CO,{EXD,RA,IND,MtC}->{OV_2,OV_1}->CO};
}; % Beispiel 4 Ende
\end{document}