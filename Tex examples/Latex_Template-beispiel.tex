\documentclass[11pt]{standalone}
\usepackage{tikz}
\usetikzlibrary{arrows, fit, positioning, shapes, graphs, quotes, calc}

\tikzset{>=latex,%
  every node/.style={circle, draw, minimum size=6mm, text width=5.5mm, align=center, inner sep = 0pt},%
  aux/.style={circle, draw, fill=black, minimum size=0.5mm, text width=0.5mm},  
  neg/.style={diamond, draw, minimum size=1mm, text width=1mm}}

\begin{document}
\def\hilightsource#1{\draw[densely dashed,-latex] ($(#1)-(8mm,0)$) -> ($(#1)-(3mm,0)$);%
  \fill [green, opacity=.25] (#1) circle [radius=3mm]; }%
\def\hilighttarget#1{\draw[densely dashed,-latex] ($(#1)+(3mm,0)$) -> ($(#1)+(8mm,0)$);%
  \fill [red, opacity=.25] (#1) circle [radius=3mm]; }%
% Laengenkonstanten
\def\LNeg{0.03cm}% Abstand des Negationssymbols von der Ausgangsnode
\def\LKonj{-0.5cm}% Abstand des Konjuntions-Knotenpunktes von der Zielnode
\def\LvDist{0.8cm}% vertikaler Abstand uebereinander befindlicher Nodes der selben Ebene
\def\LhDist{1cm}% horizontaler Abstand nebeneinander befindlicher Nodes
\def\iLvDist{1.4cm}% vertikaler Abstand zwischen benachbarten Ebenen
\begin{tikzpicture}
\node (a) {a};
\node[right= \LhDist of a] (b) {b};
\node[above= \LvDist of a] (c) {c};
\node[above= \LvDist of b] (d) {d};
\node[above= \LvDist of d] (e) {e};

\node[above= {\LvDist + \iLvDist} of c] (f1) {f$_1$};

% Disjunktion
  \draw [->] (a.east) -- (b.west);
  \draw [->] (c.east) -- (b.west);

% Konjunktion
  % Schnittpunkt der Konjunkten
  \node[aux] (daux) at ([xshift=\LKonj]d.west) {};

  % Teilpfeile von den Konjunkten zum Schnittpunkt
  \draw[in=170,out=-10] (c) to (daux);
  \draw[in=170,out=-10] (a) to (daux);

  % Pfeil vom Schnittpunkt
  \draw[->] (daux) -- (d);
    
% Negation
  \node[neg] (cneg) at ([xshift=\LNeg]c.north east) {};
  \draw[->] (cneg) -- (e);

% Konstitutionsbeziehung links
  \draw[densely dotted,opacity=0.6,in=180,out=180] (c.north) to (f1.south);
  
% Konstitutionsbeziehung rechts
  \draw[densely dotted,opacity=0.6,in=0,out=0] (d.north) to (f1.south);

\hilightsource{a};
\hilightsource{c};
\hilighttarget{b};
\hilighttarget{d};
\hilighttarget{e};
% Include formula from file
%\VAR{tex_formula}
\end{tikzpicture}
\end{document}