\documentclass[11pt,varwidth]{standalone}
\usepackage{tikz}
\usetikzlibrary{arrows, fit, positioning, shapes, graphs, quotes, calc}

\begin{document}
\def\hilightsource#1{\draw[densely dashed,-latex] ($(#1)-(8mm,0)$) -> ($(#1)-(3mm,0)$);%
  \fill [green, opacity=.25] (#1) circle [radius=3mm]; }%
\def\hilighttarget#1{\draw[densely dashed,-latex] ($(#1)+(3mm,0)$) -> ($(#1)+(8mm,0)$);%
  \fill [red, opacity=.25] (#1) circle [radius=3mm]; }%
% Laengenkonstanten
\def\LNeg{0.03cm}% Abstand des Negationssymbols von der Ausgangsnode
\def\LConj{-0.5cm}% Abstand des Konjuntions-Knotenpunktes von der Zielnode
\def\hDisjConj{0.9cm}% horizontale Positionierung des Konjuntions-Knotenpunktes innerhalb einer Disjunktion von der ersten Ausgangsnode
\def\vDisjConj{0.2cm}% vertikale Positionierung des Konjuntions-Knotenpunktes innerhalb einer Disjunktion von der ersten Ausgangsnode
\def\tDisjConj{0.3cm}% Verschiebung eines Konjunktionsknotenpunktes, falls mehrere innerhalb von Disjunktionen am gleichen Faktor zusammenfallen
\def\LvDist{0.8cm}% vertikaler Abstand uebereinander befindlicher Nodes der selben Ebene
\def\LhDist{1.2cm}% horizontaler Abstand nebeneinander befindlicher Nodes
\def\iLvDist{1.4cm}% vertikaler Abstand zwischen benachbarten Ebenen
\def\HeightNode{0.6cm}% Hoehe einer Node
\begin{tikzpicture}[>=latex,%
  every node/.style={circle, draw, minimum size=6mm, text width=5.5mm, align=center, inner sep = 0pt},% gilt fuer alle normalen Nodes
  aux/.style={circle, draw, fill=black, minimum size=0.5mm, text width=0.5mm},% Hilfsnodes an Konjunktionsschnittpunkten
  neg/.style={diamond, draw, minimum size=1mm, text width=1mm},% Hilfsnode Negation
  conjunctonsegment/.style={in=170,out=-10},% gewoelbte Verbindungen um konjunktive Verbindungen einfach von disjunktiven unterscheiden zu koennen
  crelationleft/.style={->,densely dotted,opacity=0.6,in=200,out=120},% Linienstil linksumschliessende Konstitutionsbeziehung
  crelationright/.style={->,densely dotted,opacity=0.6,in=-20,out=60},% Linienstil rechtsumschliessende Konstitutionsbeziehung
  crelationstraight/.style={->,densely dotted,opacity=0.6}]% Linienstil geradlinige Konstitutionsbeziehung
% Begin formula from file
% Platzierung der Nodes
% Faktoren der Ebene 0:
% Kausalordnung 0:
\node (D) {D};
\hilightsource{D};
\node[above= \LvDist of D] (E) {E};
\hilightsource{E};
\node[above= \LvDist of E] (G) {G};
\hilightsource{G};
\node[above= \LvDist of G] (H) {H};
\hilightsource{H};
% Kausalordnung 1:
\node[right= \LhDist of D] (F) {F};
\node[above= \LvDist of F] (I) {I};
% Kausalordnung 2:
\node[right= \LhDist of F] (J) {J};
\hilighttarget{J};
% Faktoren der Ebene 1:
% Kausalordnung 0:
\node[above= {3*\LvDist + 3* \HeightNode  + \iLvDist} of D] (A) {A};
\hilightsource{A};
\node[above= \LvDist of A] (B) {B};
\hilightsource{B};
% Kausalordnung 1:
\node[right= \LhDist of A] (C) {C};
\hilighttarget{C};

% causal relations
% of level 0
% formula: D*E <-> F
% Treffpunkt der Konjunkten
\node[aux] (Faux) at ([xshift=\LConj]F.west) {};
% Teilpfeile von den Konjunkten zum Schnittpunkt
\draw[conjunctonsegment] (D) to (Faux);
\draw[conjunctonsegment] (E) to (Faux);
% Pfeil vom Schnittpunkt zum Zielfaktor
\draw[->] (Faux) -- (F);

% formula: G*H <-> I
% Treffpunkt der Konjunkten
\node[aux] (Iaux) at ([xshift=\LConj]I.west) {};
% Teilpfeile von den Konjunkten zum Schnittpunkt
\draw[conjunctonsegment] (G) to (Iaux);
\draw[conjunctonsegment] (H) to (Iaux);
% Pfeil vom Schnittpunkt zum Zielfaktor
\draw[->] (Iaux) -- (I);

% formula: F*I <-> J
% Treffpunkt der Konjunkten
\node[aux] (Jaux) at ([xshift=\LConj]J.west) {};
% Teilpfeile von den Konjunkten zum Schnittpunkt
\draw[conjunctonsegment] (F) to (Jaux);
\draw[conjunctonsegment] (I) to (Jaux);
% Pfeil vom Schnittpunkt zum Zielfaktor
\draw[->] (Jaux) -- (J);

% of level 1
% formula: A*B <-> C
% Treffpunkt der Konjunkten
\node[aux] (Caux) at ([xshift=\LConj]C.west) {};
% Teilpfeile von den Konjunkten zum Schnittpunkt
\draw[conjunctonsegment] (A) to (Caux);
\draw[conjunctonsegment] (B) to (Caux);
% Pfeil vom Schnittpunkt zum Zielfaktor
\draw[->] (Caux) -- (C);


% constitution relations
% formula: F <-> A
\draw[crelationright] (F.north east) to (A.south);

% formula: D*E <-> A
\draw[crelationleft] (D.north west) to (A.south);
\draw[crelationleft] (E.north west) to (A.south);

% formula: I <-> B
\draw[crelationright] (I.north east) to (B.south);

% formula: G*H <-> B
\draw[crelationleft] (G.north west) to (B.south);
\draw[crelationleft] (H.north west) to (B.south);

% formula: J <-> C
\draw[crelationstraight] (J.north) to (C.south);


% End formula from file
\end{tikzpicture}
\end{document}