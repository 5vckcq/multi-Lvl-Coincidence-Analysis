\documentclass[multi=page,crop,11pt,varwidth,border=2pt]{standalone}
\usepackage{xcolor}
\usepackage{tikz}
\usepackage{graphicx} % for scale box
\usetikzlibrary{arrows, fit, positioning, shapes, graphs, quotes, calc, bbox}

\begin{document}
% define colors (ColorBrewer scheme Paired-12
\definecolor{color0}{RGB}{166,206,227}
\definecolor{color1}{RGB}{31,120,180}
\definecolor{color2}{RGB}{178,223,138}
\definecolor{color3}{RGB}{51,160,44}
\definecolor{color4}{RGB}{251,154,153}
\definecolor{color5}{RGB}{227,26,28}
\definecolor{color6}{RGB}{253,191,111}
\definecolor{color7}{RGB}{255,127,0}
\definecolor{color8}{RGB}{202,178,214}
\definecolor{color9}{RGB}{106,61,154}
\definecolor{color10}{RGB}{245,227,101} % original value is poorly legible, therefore slightly modified
\definecolor{color11}{RGB}{177,89,40}

\tikzset{every node/.style={circle, draw, thick, minimum size=6mm, text width=5.5mm, align=center, inner sep = 0pt}}% definition for normal nodes = nodes of causal factors
\tikzstyle{aux} = [circle, draw, fill, minimum size=0.5mm, text width=0.5mm]% auxiliary nodes for conjunction junctions
\tikzstyle{neg} = [diamond, draw, minimum size=1mm, text width=1mm]% auxiliary nodes for negations
\tikzstyle{conjunctonsegment} = [in=170,out=-10]% conjuctive connection lines are curved to easily distinguish them from disjunctions
\tikzstyle{crelationleft} = [->,densely dotted,in=200,out=120]% style for the leftside constitution relations
\tikzstyle{crelationright} = [->,densely dotted,in=-20,out=60]% style for the rightside constitution relations
\tikzstyle{crelationstraight} = [->,densely dotted]% style for single constitution relations


\def\hilightsource#1{\draw[densely dashed,-latex] ($(#1)-(8mm,0)$) -> ($(#1)-(3mm,0)$)}%
\def\hilighttarget#1{\draw[densely dashed,-latex] ($(#1)+(3mm,0)$) -> ($(#1)+(8mm,0)$)}%
% definition of constants
\def\LNeg{0.03cm}% distance of the negation symbol to the corresponding node of the causal factor
\def\LConj{-0.5cm}% distance of the conjunction junction to the target node
\def\hDisjConj{0.9cm}% horizontal position of conjunction junction of a disjunction relative to the first source node (normal, non-circular case)
\def\hcDisjConj{-0.3cm}% horizontal position of conjunction junction of a disjunction relative to the first source node (circular case, source and target node are horizontalle aligned)
\def\vDisjConj{0.2cm}% vertical position of conjunction junction of a disjunction relative to the first source node
\def\tDisjConj{0.3cm}% shift of  conjunction junction in case that it would overlap with another junction
\def\LvDist{0.8cm}% vertical distance between nodes of the same level and causal order
\def\LhDist{2cm}% horizontal distance between nodes of the same level and subsequent causal orders
\def\iLvDist{1.4cm}% vertical distance between neighbouring levels
\def\HeightNode{0.6cm}% height of nodes
\def\LabelDist{-2mm}% vertical distance of labels from vertices


\begin{page}
\centering solution no.~1/6\\[3mm]

\begin{tikzpicture}[>=latex,%
  %bezier bounding box=true,% avoid excessive white spaces around bended lines [comment this line out if compiling takes too long]  
  ]%
% Begin formula from file
% placement of the nodes
% factors of level 0:
% causal order 0:
\node[draw=color0, text=black]  (H) {$H$};
\hilightsource{H};
\node[draw=color0, text=black] [above= \LvDist of H] (G) {$G$};
\hilightsource{G};
\node[draw=color1, text=black] [above= \LvDist of G] (E) {$E$};
\hilightsource{E};
\node[draw=color1, text=black] [above= \LvDist of E] (D) {$D$};
\hilightsource{D};
% causal order 1:
\node[draw=color0, text=black] [right= \LhDist of H] (I) {$I$};
\node[draw=color1, text=black] [above= \LvDist of I] (F) {$F$};
\hilighttarget{F};
% causal order 2:
\node[draw=color2, text=black] [right= \LhDist of I] (J) {$J$};
\hilighttarget{J};
% factors of level 1:
% causal order 0:
\node[draw=black, text=color0] [above= {4*\LvDist + 2* \HeightNode  + \iLvDist} of H] (B) {$B$};
\hilightsource{B};
\node[draw=black, text=color1] [above= \LvDist of B] (A) {$A$};
\hilightsource{A};
% causal order 1:
\node[draw=black, text=color2] [right= \LhDist of B] (C) {$C$};
\hilighttarget{C};

% causal relations
% of level 0
% formula: E*I + ~D*G*H <-> J
% complex disjunction
% junction of the conjuncts
\node[aux, black] (EIJaux) at ([xshift=\hDisjConj, yshift=\vDisjConj]I.east) {};
% partial arrows from the conjuncts to the junction
\draw[conjunctonsegment, black] (E.east) to (EIJaux);
\draw[conjunctonsegment, black] (I.east) to (EIJaux);
% arrow from junction to target factor
\draw[->, black] (EIJaux) -- (J.west) node[draw=none,text=black,fill=none,font=\tiny,pos=0,sloped,above=\LabelDist] {\scalebox{.3}{$E\cdot I$}};
% complex disjunction
% junction of the conjuncts
\node[aux, black] (HGDJaux) at ([xshift=\hDisjConj, yshift=\vDisjConj]H.east) {};
% partial arrows from the conjuncts to the junction
\node[neg, color1] (Dneg) at ([xshift=\LNeg]D.south east) {};
\draw[conjunctonsegment, black] (Dneg) to (HGDJaux);
\draw[conjunctonsegment, black] (G.east) to (HGDJaux);
\draw[conjunctonsegment, black] (H.east) to (HGDJaux);
% arrow from junction to target factor
\draw[->, black] (HGDJaux) -- (J.west) node[draw=none,text=black,fill=none,font=\tiny,pos=0,sloped,above=\LabelDist] {\scalebox{.3}{$\neg D\cdot G\cdot H$}};


% formula: ~D + E <-> F
% negated disjunct
\node[neg, color1] (Dneg) at ([xshift=\LNeg]D.south east) {};
\draw[->, color1] (Dneg) to (F.west);
% simple disjunction with shifted starting point
\draw[->, color1] (E.north east) to (F.west);


% formula: G*H <-> I
% junction of the conjuncts
\node[aux, color0] (Iaux) at ([xshift=\LConj]I.west) {};
% partial arrows from the conjuncts to the junction
\draw[conjunctonsegment, color0] (H.east) to (Iaux);
\draw[conjunctonsegment, color0] (G.east) to (Iaux);
% arrow from junction to target factor
\draw[->, color0] (Iaux) -- (I) node[draw=none, text=black, fill=none, font=\tiny, above=\LabelDist, pos=0, sloped] {\scalebox{.3}{$G\cdot H$}};


% of level 1
% formula: A*B <-> C
% junction of the conjuncts
\node[aux, black] (Caux) at ([xshift=\LConj]C.west) {};
% partial arrows from the conjuncts to the junction
\draw[conjunctonsegment, black] (B.east) to (Caux);
\draw[conjunctonsegment, black] (A.east) to (Caux);
% arrow from junction to target factor
\draw[->, black] (Caux) -- (C) node[draw=none, text=black, fill=none, font=\tiny, above=\LabelDist, pos=0, sloped] {\scalebox{.3}{$A\cdot B$}};



% constitution relations
% formula: I <-> B
\draw[crelationright, color0] (I.north east) to (B.south);

% formula: H <-> B
\draw[crelationleft, color0] (H.north west) to (B.south);

% formula: G <-> B
\draw[crelationleft, color0] (G.north west) to (B.south);

% formula: F <-> A
\draw[crelationright, color1] (F.north east) to (A.south);

% formula: E <-> A
\draw[crelationleft, color1] (E.north west) to (A.south);

% formula: D <-> A
\draw[crelationleft, color1] (D.north west) to (A.south);

% formula: J <-> C
\draw[crelationstraight, color2] (J.north) to (C.south);


% End formula from file
\end{tikzpicture}\\[4mm]
\tiny $(E \cdot I + \neg D \cdot G \cdot H\leftrightarrow J)\cdot(\neg D + E\leftrightarrow F)\cdot(G \cdot H\leftrightarrow I)\cdot(A \cdot B\leftrightarrow C)$
\end{page}

\begin{page}
\centering solution no.~2/6\\[3mm]

\begin{tikzpicture}[>=latex,%
  %bezier bounding box=true,% avoid excessive white spaces around bended lines [comment this line out if compiling takes too long]  
  ]%
% Begin formula from file
% placement of the nodes
% factors of level 0:
% causal order 0:
\node[draw=color0, text=black]  (H) {$H$};
\hilightsource{H};
\node[draw=color0, text=black] [above= \LvDist of H] (G) {$G$};
\hilightsource{G};
\node[draw=color1, text=black] [above= \LvDist of G] (E) {$E$};
\hilightsource{E};
\node[draw=color1, text=black] [above= \LvDist of E] (D) {$D$};
\hilightsource{D};
% causal order 1:
\node[draw=color0, text=black] [right= \LhDist of H] (I) {$I$};
\hilighttarget{I};
\node[draw=color1, text=black] [above= \LvDist of I] (F) {$F$};
% causal order 2:
\node[draw=color2, text=black] [right= \LhDist of I] (J) {$J$};
\hilighttarget{J};
% factors of level 1:
% causal order 0:
\node[draw=black, text=color0] [above= {4*\LvDist + 2* \HeightNode  + \iLvDist} of H] (B) {$B$};
\hilightsource{B};
\node[draw=black, text=color1] [above= \LvDist of B] (A) {$A$};
\hilightsource{A};
% causal order 1:
\node[draw=black, text=color2] [right= \LhDist of B] (C) {$C$};
\hilighttarget{C};

% causal relations
% of level 0
% formula: F*G*H <-> J
% junction of the conjuncts
\node[aux, black] (Jaux) at ([xshift=\LConj]J.west) {};
% partial arrows from the conjuncts to the junction
\draw[conjunctonsegment, black] (H.east) to (Jaux);
\draw[conjunctonsegment, black] (G.east) to (Jaux);
\draw[conjunctonsegment, black] (F.east) to (Jaux);
% arrow from junction to target factor
\draw[->, black] (Jaux) -- (J) node[draw=none, text=black, fill=none, font=\tiny, above=\LabelDist, pos=0, sloped] {\scalebox{.3}{$F\cdot G\cdot H$}};


% formula: ~D + E <-> F
% negated disjunct
\node[neg, color1] (Dneg) at ([xshift=\LNeg]D.south east) {};
\draw[->, color1] (Dneg) to (F.west);
% simple disjunction with shifted starting point
\draw[->, color1] (E.north east) to (F.west);


% formula: G*H <-> I
% junction of the conjuncts
\node[aux, color0] (Iaux) at ([xshift=\LConj]I.west) {};
% partial arrows from the conjuncts to the junction
\draw[conjunctonsegment, color0] (H.east) to (Iaux);
\draw[conjunctonsegment, color0] (G.east) to (Iaux);
% arrow from junction to target factor
\draw[->, color0] (Iaux) -- (I) node[draw=none, text=black, fill=none, font=\tiny, above=\LabelDist, pos=0, sloped] {\scalebox{.3}{$G\cdot H$}};


% of level 1
% formula: A*B <-> C
% junction of the conjuncts
\node[aux, black] (Caux) at ([xshift=\LConj]C.west) {};
% partial arrows from the conjuncts to the junction
\draw[conjunctonsegment, black] (B.east) to (Caux);
\draw[conjunctonsegment, black] (A.east) to (Caux);
% arrow from junction to target factor
\draw[->, black] (Caux) -- (C) node[draw=none, text=black, fill=none, font=\tiny, above=\LabelDist, pos=0, sloped] {\scalebox{.3}{$A\cdot B$}};



% constitution relations
% formula: I <-> B
\draw[crelationright, color0] (I.north east) to (B.south);

% formula: H <-> B
\draw[crelationleft, color0] (H.north west) to (B.south);

% formula: G <-> B
\draw[crelationleft, color0] (G.north west) to (B.south);

% formula: F <-> A
\draw[crelationright, color1] (F.north east) to (A.south);

% formula: E <-> A
\draw[crelationleft, color1] (E.north west) to (A.south);

% formula: D <-> A
\draw[crelationleft, color1] (D.north west) to (A.south);

% formula: J <-> C
\draw[crelationstraight, color2] (J.north) to (C.south);


% End formula from file
\end{tikzpicture}\\[4mm]
\tiny $(F \cdot G \cdot H\leftrightarrow J)\cdot(\neg D + E\leftrightarrow F)\cdot(G \cdot H\leftrightarrow I)\cdot(A \cdot B\leftrightarrow C)$
\end{page}

\begin{page}
\centering solution no.~3/6\\[3mm]

\begin{tikzpicture}[>=latex,%
  %bezier bounding box=true,% avoid excessive white spaces around bended lines [comment this line out if compiling takes too long]  
  ]%
% Begin formula from file
% placement of the nodes
% factors of level 0:
% causal order 0:
\node[draw=color0, text=black]  (H) {$H$};
\hilightsource{H};
\node[draw=color0, text=black] [above= \LvDist of H] (G) {$G$};
\hilightsource{G};
\node[draw=color1, text=black] [above= \LvDist of G] (E) {$E$};
\hilightsource{E};
\node[draw=color1, text=black] [above= \LvDist of E] (D) {$D$};
\hilightsource{D};
% causal order 1:
\node[draw=color0, text=black] [right= \LhDist of H] (I) {$I$};
\node[draw=color1, text=black] [above= \LvDist of I] (F) {$F$};
% causal order 2:
\node[draw=color2, text=black] [right= \LhDist of I] (J) {$J$};
\hilighttarget{J};
% factors of level 1:
% causal order 0:
\node[draw=black, text=color0] [above= {4*\LvDist + 2* \HeightNode  + \iLvDist} of H] (B) {$B$};
\hilightsource{B};
\node[draw=black, text=color1] [above= \LvDist of B] (A) {$A$};
\hilightsource{A};
% causal order 1:
\node[draw=black, text=color2] [right= \LhDist of B] (C) {$C$};
\hilighttarget{C};

% causal relations
% of level 0
% formula: F*I <-> J
% junction of the conjuncts
\node[aux, black] (Jaux) at ([xshift=\LConj]J.west) {};
% partial arrows from the conjuncts to the junction
\draw[conjunctonsegment, black] (I.east) to (Jaux);
\draw[conjunctonsegment, black] (F.east) to (Jaux);
% arrow from junction to target factor
\draw[->, black] (Jaux) -- (J) node[draw=none, text=black, fill=none, font=\tiny, above=\LabelDist, pos=0, sloped] {\scalebox{.3}{$F\cdot I$}};


% formula: ~D + E <-> F
% negated disjunct
\node[neg, color1] (Dneg) at ([xshift=\LNeg]D.south east) {};
\draw[->, color1] (Dneg) to (F.west);
% simple disjunction with shifted starting point
\draw[->, color1] (E.north east) to (F.west);


% formula: G*H <-> I
% junction of the conjuncts
\node[aux, color0] (Iaux) at ([xshift=\LConj]I.west) {};
% partial arrows from the conjuncts to the junction
\draw[conjunctonsegment, color0] (H.east) to (Iaux);
\draw[conjunctonsegment, color0] (G.east) to (Iaux);
% arrow from junction to target factor
\draw[->, color0] (Iaux) -- (I) node[draw=none, text=black, fill=none, font=\tiny, above=\LabelDist, pos=0, sloped] {\scalebox{.3}{$G\cdot H$}};


% of level 1
% formula: A*B <-> C
% junction of the conjuncts
\node[aux, black] (Caux) at ([xshift=\LConj]C.west) {};
% partial arrows from the conjuncts to the junction
\draw[conjunctonsegment, black] (B.east) to (Caux);
\draw[conjunctonsegment, black] (A.east) to (Caux);
% arrow from junction to target factor
\draw[->, black] (Caux) -- (C) node[draw=none, text=black, fill=none, font=\tiny, above=\LabelDist, pos=0, sloped] {\scalebox{.3}{$A\cdot B$}};



% constitution relations
% formula: I <-> B
\draw[crelationright, color0] (I.north east) to (B.south);

% formula: H <-> B
\draw[crelationleft, color0] (H.north west) to (B.south);

% formula: G <-> B
\draw[crelationleft, color0] (G.north west) to (B.south);

% formula: F <-> A
\draw[crelationright, color1] (F.north east) to (A.south);

% formula: E <-> A
\draw[crelationleft, color1] (E.north west) to (A.south);

% formula: D <-> A
\draw[crelationleft, color1] (D.north west) to (A.south);

% formula: J <-> C
\draw[crelationstraight, color2] (J.north) to (C.south);


% End formula from file
\end{tikzpicture}\\[4mm]
\tiny $(F \cdot I\leftrightarrow J)\cdot(\neg D + E\leftrightarrow F)\cdot(G \cdot H\leftrightarrow I)\cdot(A \cdot B\leftrightarrow C)$
\end{page}

\begin{page}
\centering solution no.~4/6\\[3mm]

\begin{tikzpicture}[>=latex,%
  %bezier bounding box=true,% avoid excessive white spaces around bended lines [comment this line out if compiling takes too long]  
  ]%
% Begin formula from file
% placement of the nodes
% factors of level 0:
% causal order 0:
\node[draw=color0, text=black]  (H) {$H$};
\hilightsource{H};
\node[draw=color0, text=black] [above= \LvDist of H] (G) {$G$};
\hilightsource{G};
\node[draw=color1, text=black] [above= \LvDist of G] (E) {$E$};
\hilightsource{E};
\node[draw=color1, text=black] [above= \LvDist of E] (D) {$D$};
\hilightsource{D};
% causal order 1:
\node[draw=color2, text=black] [right= \LhDist of H] (J) {$J$};
\hilighttarget{J};
\node[draw=color0, text=black] [above= \LvDist of J] (I) {$I$};
\hilighttarget{I};
\node[draw=color1, text=black] [above= \LvDist of I] (F) {$F$};
\hilighttarget{F};
% factors of level 1:
% causal order 0:
\node[draw=black, text=color0] [above= {4*\LvDist + 2* \HeightNode  + \iLvDist} of H] (B) {$B$};
\hilightsource{B};
\node[draw=black, text=color1] [above= \LvDist of B] (A) {$A$};
\hilightsource{A};
% causal order 1:
\node[draw=black, text=color2] [right= \LhDist of B] (C) {$C$};
\hilighttarget{C};

% causal relations
% of level 0
% formula: ~D*G*H + E*G*H <-> J
% complex disjunction
% junction of the conjuncts
\node[aux, black] (HGDJaux) at ([xshift=\hDisjConj, yshift=\vDisjConj]H.east) {};
% partial arrows from the conjuncts to the junction
\node[neg, color1] (Dneg) at ([xshift=\LNeg]D.south east) {};
\draw[conjunctonsegment, black] (Dneg) to (HGDJaux);
\draw[conjunctonsegment, black] (G.east) to (HGDJaux);
\draw[conjunctonsegment, black] (H.east) to (HGDJaux);
% arrow from junction to target factor
\draw[->, black] (HGDJaux) -- (J.west) node[draw=none,text=black,fill=none,font=\tiny,pos=0,sloped,above=\LabelDist] {\scalebox{.3}{$\neg D\cdot G\cdot H$}};
% complex disjunction
% junction of the conjuncts
\node[aux, black] (HGEJaux) at ([xshift=\hDisjConj, yshift={\vDisjConj + 1*\tDisjConj}]H.east) {};
% partial arrows from the conjuncts to the junction
\draw[conjunctonsegment, black] (E.east) to (HGEJaux);
\draw[conjunctonsegment, black] (G.east) to (HGEJaux);
\draw[conjunctonsegment, black] (H.east) to (HGEJaux);
% arrow from junction to target factor
\draw[->, black] (HGEJaux) -- (J.west) node[draw=none,text=black,fill=none,font=\tiny,pos=0,sloped,above=\LabelDist] {\scalebox{.3}{$E\cdot G\cdot H$}};


% formula: ~D + E <-> F
% negated disjunct
\node[neg, color1] (Dneg) at ([xshift=\LNeg]D.south east) {};
\draw[->, color1] (Dneg) to (F.west);
% simple disjunction with shifted starting point
\draw[->, color1] (E.north east) to (F.west);


% formula: G*H <-> I
% junction of the conjuncts
\node[aux, color0] (Iaux) at ([xshift=\LConj]I.west) {};
% partial arrows from the conjuncts to the junction
\draw[conjunctonsegment, color0] (H.east) to (Iaux);
\draw[conjunctonsegment, color0] (G.east) to (Iaux);
% arrow from junction to target factor
\draw[->, color0] (Iaux) -- (I) node[draw=none, text=black, fill=none, font=\tiny, above=\LabelDist, pos=0, sloped] {\scalebox{.3}{$G\cdot H$}};


% of level 1
% formula: A*B <-> C
% junction of the conjuncts
\node[aux, black] (Caux) at ([xshift=\LConj]C.west) {};
% partial arrows from the conjuncts to the junction
\draw[conjunctonsegment, black] (B.east) to (Caux);
\draw[conjunctonsegment, black] (A.east) to (Caux);
% arrow from junction to target factor
\draw[->, black] (Caux) -- (C) node[draw=none, text=black, fill=none, font=\tiny, above=\LabelDist, pos=0, sloped] {\scalebox{.3}{$A\cdot B$}};



% constitution relations
% formula: I <-> B
\draw[crelationright, color0] (I.north east) to (B.south);

% formula: H <-> B
\draw[crelationleft, color0] (H.north west) to (B.south);

% formula: G <-> B
\draw[crelationleft, color0] (G.north west) to (B.south);

% formula: F <-> A
\draw[crelationright, color1] (F.north east) to (A.south);

% formula: E <-> A
\draw[crelationleft, color1] (E.north west) to (A.south);

% formula: D <-> A
\draw[crelationleft, color1] (D.north west) to (A.south);

% formula: J <-> C
\draw[crelationstraight, color2] (J.north) to (C.south);


% End formula from file
\end{tikzpicture}\\[4mm]
\tiny $(\neg D \cdot G \cdot H + E \cdot G \cdot H\leftrightarrow J)\cdot(\neg D + E\leftrightarrow F)\cdot(G \cdot H\leftrightarrow I)\cdot(A \cdot B\leftrightarrow C)$
\end{page}

\begin{page}
\centering solution no.~5/6\\[3mm]

\begin{tikzpicture}[>=latex,%
  %bezier bounding box=true,% avoid excessive white spaces around bended lines [comment this line out if compiling takes too long]  
  ]%
% Begin formula from file
% placement of the nodes
% factors of level 0:
% causal order 0:
\node[draw=color0, text=black]  (H) {$H$};
\hilightsource{H};
\node[draw=color0, text=black] [above= \LvDist of H] (G) {$G$};
\hilightsource{G};
\node[draw=color1, text=black] [above= \LvDist of G] (E) {$E$};
\hilightsource{E};
\node[draw=color1, text=black] [above= \LvDist of E] (D) {$D$};
\hilightsource{D};
% causal order 1:
\node[draw=color0, text=black] [right= \LhDist of H] (I) {$I$};
\node[draw=color1, text=black] [above= \LvDist of I] (F) {$F$};
\hilighttarget{F};
% causal order 2:
\node[draw=color2, text=black] [right= \LhDist of I] (J) {$J$};
\hilighttarget{J};
% factors of level 1:
% causal order 0:
\node[draw=black, text=color0] [above= {4*\LvDist + 2* \HeightNode  + \iLvDist} of H] (B) {$B$};
\hilightsource{B};
\node[draw=black, text=color1] [above= \LvDist of B] (A) {$A$};
\hilightsource{A};
% causal order 1:
\node[draw=black, text=color2] [right= \LhDist of B] (C) {$C$};
\hilighttarget{C};

% causal relations
% of level 0
% formula: ~D*I + E*G*H <-> J
% complex disjunction
% junction of the conjuncts
\node[aux, black] (DIJaux) at ([xshift=\hDisjConj, yshift=\vDisjConj]I.east) {};
% partial arrows from the conjuncts to the junction
\node[neg, color1] (Dneg) at ([xshift=\LNeg]D.south east) {};
\draw[conjunctonsegment, black] (Dneg) to (DIJaux);
\draw[conjunctonsegment, black] (I.east) to (DIJaux);
% arrow from junction to target factor
\draw[->, black] (DIJaux) -- (J.west) node[draw=none,text=black,fill=none,font=\tiny,pos=0,sloped,above=\LabelDist] {\scalebox{.3}{$\neg D\cdot I$}};
% complex disjunction
% junction of the conjuncts
\node[aux, black] (HGEJaux) at ([xshift=\hDisjConj, yshift=\vDisjConj]H.east) {};
% partial arrows from the conjuncts to the junction
\draw[conjunctonsegment, black] (E.east) to (HGEJaux);
\draw[conjunctonsegment, black] (G.east) to (HGEJaux);
\draw[conjunctonsegment, black] (H.east) to (HGEJaux);
% arrow from junction to target factor
\draw[->, black] (HGEJaux) -- (J.west) node[draw=none,text=black,fill=none,font=\tiny,pos=0,sloped,above=\LabelDist] {\scalebox{.3}{$E\cdot G\cdot H$}};


% formula: ~D + E <-> F
% negated disjunct
\node[neg, color1] (Dneg) at ([xshift=\LNeg]D.south east) {};
\draw[->, color1] (Dneg) to (F.west);
% simple disjunction with shifted starting point
\draw[->, color1] (E.north east) to (F.west);


% formula: G*H <-> I
% junction of the conjuncts
\node[aux, color0] (Iaux) at ([xshift=\LConj]I.west) {};
% partial arrows from the conjuncts to the junction
\draw[conjunctonsegment, color0] (H.east) to (Iaux);
\draw[conjunctonsegment, color0] (G.east) to (Iaux);
% arrow from junction to target factor
\draw[->, color0] (Iaux) -- (I) node[draw=none, text=black, fill=none, font=\tiny, above=\LabelDist, pos=0, sloped] {\scalebox{.3}{$G\cdot H$}};


% of level 1
% formula: A*B <-> C
% junction of the conjuncts
\node[aux, black] (Caux) at ([xshift=\LConj]C.west) {};
% partial arrows from the conjuncts to the junction
\draw[conjunctonsegment, black] (B.east) to (Caux);
\draw[conjunctonsegment, black] (A.east) to (Caux);
% arrow from junction to target factor
\draw[->, black] (Caux) -- (C) node[draw=none, text=black, fill=none, font=\tiny, above=\LabelDist, pos=0, sloped] {\scalebox{.3}{$A\cdot B$}};



% constitution relations
% formula: I <-> B
\draw[crelationright, color0] (I.north east) to (B.south);

% formula: H <-> B
\draw[crelationleft, color0] (H.north west) to (B.south);

% formula: G <-> B
\draw[crelationleft, color0] (G.north west) to (B.south);

% formula: F <-> A
\draw[crelationright, color1] (F.north east) to (A.south);

% formula: E <-> A
\draw[crelationleft, color1] (E.north west) to (A.south);

% formula: D <-> A
\draw[crelationleft, color1] (D.north west) to (A.south);

% formula: J <-> C
\draw[crelationstraight, color2] (J.north) to (C.south);


% End formula from file
\end{tikzpicture}\\[4mm]
\tiny $(\neg D \cdot I + E \cdot G \cdot H\leftrightarrow J)\cdot(\neg D + E\leftrightarrow F)\cdot(G \cdot H\leftrightarrow I)\cdot(A \cdot B\leftrightarrow C)$
\end{page}

\begin{page}
\centering solution no.~6/6\\[3mm]

\begin{tikzpicture}[>=latex,%
  %bezier bounding box=true,% avoid excessive white spaces around bended lines [comment this line out if compiling takes too long]  
  ]%
% Begin formula from file
% placement of the nodes
% factors of level 0:
% causal order 0:
\node[draw=color0, text=black]  (H) {$H$};
\hilightsource{H};
\node[draw=color0, text=black] [above= \LvDist of H] (G) {$G$};
\hilightsource{G};
\node[draw=color1, text=black] [above= \LvDist of G] (E) {$E$};
\hilightsource{E};
\node[draw=color1, text=black] [above= \LvDist of E] (D) {$D$};
\hilightsource{D};
% causal order 1:
\node[draw=color0, text=black] [right= \LhDist of H] (I) {$I$};
\node[draw=color1, text=black] [above= \LvDist of I] (F) {$F$};
\hilighttarget{F};
% causal order 2:
\node[draw=color2, text=black] [right= \LhDist of I] (J) {$J$};
\hilighttarget{J};
% factors of level 1:
% causal order 0:
\node[draw=black, text=color0] [above= {4*\LvDist + 2* \HeightNode  + \iLvDist} of H] (B) {$B$};
\hilightsource{B};
\node[draw=black, text=color1] [above= \LvDist of B] (A) {$A$};
\hilightsource{A};
% causal order 1:
\node[draw=black, text=color2] [right= \LhDist of B] (C) {$C$};
\hilighttarget{C};

% causal relations
% of level 0
% formula: ~D*I + E*I <-> J
% complex disjunction
% junction of the conjuncts
\node[aux, black] (DIJaux) at ([xshift=\hDisjConj, yshift=\vDisjConj]I.east) {};
% partial arrows from the conjuncts to the junction
\node[neg, color1] (Dneg) at ([xshift=\LNeg]D.south east) {};
\draw[conjunctonsegment, black] (Dneg) to (DIJaux);
\draw[conjunctonsegment, black] (I.east) to (DIJaux);
% arrow from junction to target factor
\draw[->, black] (DIJaux) -- (J.west) node[draw=none,text=black,fill=none,font=\tiny,pos=0,sloped,above=\LabelDist] {\scalebox{.3}{$\neg D\cdot I$}};
% complex disjunction
% junction of the conjuncts
\node[aux, black] (EIJaux) at ([xshift=\hDisjConj, yshift={\vDisjConj + 1*\tDisjConj}]I.east) {};
% partial arrows from the conjuncts to the junction
\draw[conjunctonsegment, black] (E.east) to (EIJaux);
\draw[conjunctonsegment, black] (I.east) to (EIJaux);
% arrow from junction to target factor
\draw[->, black] (EIJaux) -- (J.west) node[draw=none,text=black,fill=none,font=\tiny,pos=0,sloped,above=\LabelDist] {\scalebox{.3}{$E\cdot I$}};


% formula: ~D + E <-> F
% negated disjunct
\node[neg, color1] (Dneg) at ([xshift=\LNeg]D.south east) {};
\draw[->, color1] (Dneg) to (F.west);
% simple disjunction with shifted starting point
\draw[->, color1] (E.north east) to (F.west);


% formula: G*H <-> I
% junction of the conjuncts
\node[aux, color0] (Iaux) at ([xshift=\LConj]I.west) {};
% partial arrows from the conjuncts to the junction
\draw[conjunctonsegment, color0] (H.east) to (Iaux);
\draw[conjunctonsegment, color0] (G.east) to (Iaux);
% arrow from junction to target factor
\draw[->, color0] (Iaux) -- (I) node[draw=none, text=black, fill=none, font=\tiny, above=\LabelDist, pos=0, sloped] {\scalebox{.3}{$G\cdot H$}};


% of level 1
% formula: A*B <-> C
% junction of the conjuncts
\node[aux, black] (Caux) at ([xshift=\LConj]C.west) {};
% partial arrows from the conjuncts to the junction
\draw[conjunctonsegment, black] (B.east) to (Caux);
\draw[conjunctonsegment, black] (A.east) to (Caux);
% arrow from junction to target factor
\draw[->, black] (Caux) -- (C) node[draw=none, text=black, fill=none, font=\tiny, above=\LabelDist, pos=0, sloped] {\scalebox{.3}{$A\cdot B$}};



% constitution relations
% formula: I <-> B
\draw[crelationright, color0] (I.north east) to (B.south);

% formula: H <-> B
\draw[crelationleft, color0] (H.north west) to (B.south);

% formula: G <-> B
\draw[crelationleft, color0] (G.north west) to (B.south);

% formula: F <-> A
\draw[crelationright, color1] (F.north east) to (A.south);

% formula: E <-> A
\draw[crelationleft, color1] (E.north west) to (A.south);

% formula: D <-> A
\draw[crelationleft, color1] (D.north west) to (A.south);

% formula: J <-> C
\draw[crelationstraight, color2] (J.north) to (C.south);


% End formula from file
\end{tikzpicture}\\[4mm]
\tiny $(\neg D \cdot I + E \cdot I\leftrightarrow J)\cdot(\neg D + E\leftrightarrow F)\cdot(G \cdot H\leftrightarrow I)\cdot(A \cdot B\leftrightarrow C)$
\end{page}


\end{document}