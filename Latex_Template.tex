% evtl. hilfreich:
% - Subgraph (vgl. 28.10.2 in pgf Handbuch)
\documentclass[11pt]{standalone}
\usepackage{tikz}
\usetikzlibrary{arrows, backgrounds, fit, positioning, shapes, graphs, quotes, calc} %%, graphdrawing}
%%\usegdlibrary{layered}

\tikzset{>=latex}

\begin{document}
\def\hilightsource#1{\draw[densely dashed,-latex] ($(#1)-(8mm,0)$) -> ($(#1)-(3mm,0)$);
  \fill [green, opacity=.25] (#1) circle [radius=3mm]; }
\def\hilighttarget#1{\draw[densely dashed,-latex] ($(#1)+(3mm,0)$) -> ($(#1)+(8mm,0)$);
  \fill [red, opacity=.25] (#1) circle [radius=3mm]; }

\tikz[every node/.style={circle, draw, minimum size=6mm, text width=5.5mm, align=center, inner sep = 0pt},
  line join=round,
  every new -!-/.style = {width=211mm},
  every new --/.style = {dotted,thin,bend left=20},
  every new ->/.style = {densely dotted},
  every new <->/.style = {densely dotted}]
\graph[math nodes,
  grow right=20mm, branch up = 14mm, 
%layered layout,
%%grow=up, components go right,
%%level distance=18mm, sibling sep=1mm, sibling distance=12mm,
  operator=\tikzgraphforeachcolorednode{source}{\hilightsource},
  operator=\tikzgraphforeachcolorednode{target}{\hilighttarget},
]
% Include formula from file
\VAR{tex_formula}
\end{document}