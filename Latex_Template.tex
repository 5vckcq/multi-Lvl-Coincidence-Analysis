\documentclass[multi=page,crop,11pt,varwidth,border=2pt]{standalone}
\usepackage{xcolor}
\usepackage{tikz}
\usepackage{graphicx} % for scale box
\usetikzlibrary{arrows, fit, positioning, shapes, graphs, quotes, calc, bbox}

\begin{document}
% define colors (ColorBrewer scheme Paired-12
\definecolor{color0}{RGB}{166,206,227}
\definecolor{color1}{RGB}{31,120,180}
\definecolor{color2}{RGB}{178,223,138}
\definecolor{color3}{RGB}{51,160,44}
\definecolor{color4}{RGB}{251,154,153}
\definecolor{color5}{RGB}{227,26,28}
\definecolor{color6}{RGB}{253,191,111}
\definecolor{color7}{RGB}{255,127,0}
\definecolor{color8}{RGB}{202,178,214}
\definecolor{color9}{RGB}{106,61,154}
\definecolor{color10}{RGB}{245,227,101} % original value is poorly legible, therefore slightly modified
\definecolor{color11}{RGB}{177,89,40}

\tikzset{every node/.style={circle, draw, thick, minimum size=6mm, text width=5.5mm, align=center, inner sep = 0pt}}% definition for normal nodes = nodes of causal factors
\tikzstyle{aux} = [circle, draw, fill, minimum size=0.5mm, text width=0.5mm]% auxiliary nodes for conjunction junctions
\tikzstyle{neg} = [diamond, draw, minimum size=1mm, text width=1mm]% auxiliary nodes for negations
\tikzstyle{conjunctonsegment} = [in=170,out=-10]% conjuctive connection lines are curved to easily distinguish them from disjunctions
\tikzstyle{crelationleft} = [->,densely dotted,in=200,out=120]% style for the leftside constitution relations
\tikzstyle{crelationright} = [->,densely dotted,in=-20,out=60]% style for the rightside constitution relations
\tikzstyle{crelationstraight} = [->,densely dotted]% style for single constitution relations


\def\hilightsource#1{\draw[densely dashed,-latex] ($(#1)-(8mm,0)$) -> ($(#1)-(3mm,0)$)}%
\def\hilighttarget#1{\draw[densely dashed,-latex] ($(#1)+(3mm,0)$) -> ($(#1)+(8mm,0)$)}%
% definition of constants
\def\LNeg{0.03cm}% distance of the negation symbol to the corresponding node of the causal factor
\def\LConj{-0.5cm}% distance of the conjunction junction to the target node
\def\hDisjConj{0.9cm}% horizontal position of conjunction junction of a disjunction relative to the first source node (normal, non-circular case)
\def\hcDisjConj{-0.3cm}% horizontal position of conjunction junction of a disjunction relative to the first source node (circular case, source and target node are horizontalle aligned)
\def\vDisjConj{0.2cm}% vertical position of conjunction junction of a disjunction relative to the first source node
\def\tDisjConj{0.3cm}% shift of  conjunction junction in case that it would overlap with another junction
\def\LvDist{0.8cm}% vertical distance between nodes of the same level and causal order
\def\LhDist{2cm}% horizontal distance between nodes of the same level and subsequent causal orders
\def\iLvDist{1.4cm}% vertical distance between neighbouring levels
\def\HeightNode{0.6cm}% height of nodes
\def\LabelDist{-2mm}% vertical distance of labels from vertices

\BLOCK{for index, entry, subtitle in data}
\begin{page}
\centering solution no.~\VAR{index}/\VAR{maxnumber}\\[3mm]

\begin{tikzpicture}[>=latex,%
  %bezier bounding box=true,% avoid excessive white spaces around bended lines [comment this line out if compiling takes too long]  
  ]%
% Begin formula from file
\VAR{entry}
% End formula from file
\end{tikzpicture}\\[4mm]
\VAR{subtitle}
\end{page}
\BLOCK{endfor}

\end{document}