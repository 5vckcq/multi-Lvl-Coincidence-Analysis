\documentclass[11pt,varwidth]{standalone}
\usepackage{tikz}
\usetikzlibrary{arrows, fit, positioning, shapes, graphs, quotes, calc}

\begin{document}
\def\hilightsource#1{\draw[densely dashed,-latex] ($(#1)-(8mm,0)$) -> ($(#1)-(3mm,0)$);%
  \fill [green, opacity=.25] (#1) circle [radius=3mm]; }%
\def\hilighttarget#1{\draw[densely dashed,-latex] ($(#1)+(3mm,0)$) -> ($(#1)+(8mm,0)$);%
  \fill [red, opacity=.25] (#1) circle [radius=3mm]; }%
% Laengenkonstanten
\def\LNeg{0.03cm}% Abstand des Negationssymbols von der Ausgangsnode
\def\LConj{-0.5cm}% Abstand des Konjuntions-Knotenpunktes von der Zielnode
\def\hDisjConj{0.9cm}% horizontale Positionierung des Konjuntions-Knotenpunktes innerhalb einer Disjunktion von der ersten Ausgangsnode
\def\vDisjConj{0.2cm}% vertikale Positionierung des Konjuntions-Knotenpunktes innerhalb einer Disjunktion von der ersten Ausgangsnode
\def\tDisjConj{0.3cm}% Verschiebung eines Konjunktionsknotenpunktes, falls mehrere innerhalb von Disjunktionen am gleichen Faktor zusammenfallen
\def\LvDist{0.8cm}% vertikaler Abstand uebereinander befindlicher Nodes der selben Ebene
\def\LhDist{1.2cm}% horizontaler Abstand nebeneinander befindlicher Nodes
\def\iLvDist{1.4cm}% vertikaler Abstand zwischen benachbarten Ebenen
\def\HeightNode{0.6cm}% Hoehe einer Node
\begin{tikzpicture}[>=latex,%
  every node/.style={circle, draw, minimum size=6mm, text width=5.5mm, align=center, inner sep = 0pt},% gilt fuer alle normalen Nodes
  aux/.style={circle, draw, fill=black, minimum size=0.5mm, text width=0.5mm},% Hilfsnodes an Konjunktionsschnittpunkten
  neg/.style={diamond, draw, minimum size=1mm, text width=1mm},% Hilfsnode Negation
  conjunctonsegment/.style={in=170,out=-10},% gewoelbte Verbindungen um konjunktive Verbindungen einfach von disjunktiven unterscheiden zu koennen
  crelationleft/.style={->,densely dotted,opacity=0.6,in=200,out=120},% Linienstil linksumschliessende Konstitutionsbeziehung
  crelationright/.style={->,densely dotted,opacity=0.6,in=-20,out=60},% Linienstil rechtsumschliessende Konstitutionsbeziehung
  crelationstraight/.style={->,densely dotted,opacity=0.6}]% Linienstil geradlinige Konstitutionsbeziehung
% Begin formula from file
\VAR{tex_formula}
% End formula from file
\end{tikzpicture}
\end{document}